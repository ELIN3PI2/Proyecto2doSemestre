\documentclass{article}
\usepackage{graphicx}

\begin{document}
\begin{titlepage}
    \centering
    \Huge{\bfseries{PROYECTO DE \\
    COMUNICACIÓN\\
    -\\
    PROGRAMACIÓN}}\\
    \vspace{3cm}
    \large{\bfseries{INFORME}}\\
    \vspace{1.5cm}
    \large{\bfseries{Infraestructura Energética en un Contexto de Desafíos}}
    
    \vspace{1.5cm}
    \underline{Equipo B:}\\
    Katherine Rodríguez Rodríguez\\
    Marian Aguilar Tavier\\
    Xavier Suárez Arcia\\
    Jessica de la Caridad Pérez Rodríguez\\
    Vicente Cao Tarrero
\end{titlepage}
En el sistema electroenergético nacional cubano el 40.6 porciento de la capacidad de generación se produce en centrales termoeléctricas, el 21.7 porciento con motores a fuel oil, y el 21.9 porciento con motores a diésel. Estas dos últimas tecnologías, en los emplazamientos de generación distribuida instalados en todas las provincias del país.\\
Si hablamos de centrales termoeléctricas, la vida útil de las mismas es de entre 30-50 años. Sin embargo, en el caso de las 8 termoeléctricas que constituyen la base de la generación de la energía cubana, todas tienes más de 30 años de explotación excepto los dos bloques de la Felton con 25 y 21 años sincronizados. Y siete de ellos acumulan más de 40 años operando.
(www.CubaDebate.cu)\\ 

\vspace{0.5cm}

Debido a esta situación se producen varias veces al año la salida de una gran parte e incluso de todos los bloques de una misma termoeléctrica, a causa de trabajos de mantenimiento unido a condiciones climáticas y de otras índoles. Sin embargo, los trabajadores de la empresa eléctrica con la ayuda del Estado se las arreglan para dar solución en el menor tiempo posible a las fallas y problemas que se presentan.\\
Una de las termoeléctricas con mayor afectación es la CTE Antonio Guiteras, ubicada en la provincia de Matanzas y que cuenta con la mayor unidad generadora del país con una potencia de 317 MW de tecnología francesa, este bloque consume crudo nacional que recibe por oleoducto y es el más eficiente del país.(información CubaDebate).\\

\vspace{0.5cm}

Acercándonos un poco más a la historia de la CTE Antonio Guiteras en los últimos años.\\
\begin{center}
    \underline{2022: Año de grandes desafíos para la Guiteras.}
\end{center}

En el año 2022, los problemas de generación eléctrica iniciados o agudizados a partir del 2021, se hicieron notar en la CTE Antonio Guiteras.\\
El 24 de mayo, un rayo impactó en la central termoeléctrica y la dejó fuera de servicio, provocando que se tensara aún más el déficit de generación del Sistema Electronergético Nacional. Esto se evidencia en la gran demanda que se generó, anteriormente había bastante disponibilidad de corriente eléctrica, la que poco a poco fue decayendo. Rápidamente aumenta los MW indisponibles por averías, como era de esperarse.\\
\vspace{1cm}
\begin{center}
   \includegraphics[width= 1.2\textwidth]{1} \\
\end{center}

\vspace{1cm}

Por la zona occidental la madrugada del 27 de septiembre, dejó un saldo de tres fallecidos y daños materiales cuantiosos, el huracán Ian. Cinco días después de su visita, aún persisten las afectaciones al Sistema Electroenérgetico Nacional (SEN), que sufrió un colapso y dejó toda la isla a oscuras.\\

\vspace{0.5cm}
\begin{center}
   \includegraphics[width= 1.2\textwidth]{4} \\
\end{center}

\vspace{0.5cm}
\begin{center}
    \includegraphics*[width= 1.2\textwidth]{5}\\
\end{center}


\vspace{0.5cm}
\begin{center}
    \includegraphics*[width= 1.2\textwidth]{6}\\
\end{center}

\vspace{0.5cm}

Este huracán no pudo ser enfrentado con todas las condiciones pues días antes ya se encontraban varias termoeléctricas fuera de servicio, lo que esto lleva también a que haya un gran déficit de la electricidad.\\

\vspace{0.5cm}

De igual forma como consecuencia de lo ocurrido en la zona industrial de Matanzas, sale de servicio por déficit de agua en agosto de 2022.\\
El 4 de noviembre, la CTE se desacopló del SEN de forma inesperada para ejecutar más de 320 tareas de mantinimiento e incrementar ligeramente la potencia hasta 240 MW.(www.CubaDebate.cu)\\

\vspace{1cm}
\begin{center}
    \includegraphics[width= 1.2\textwidth]{2}\\
\end{center}

\vspace{1cm}

Se puede apreciar cómo en ese mes la afectación se mantuvo bastante alta. En la gráfica la región azul hace referencia al año 2022 y la blanca al año 2023. \\

\vspace{0.5cm}
\begin{center}
   \includegraphics[width= 1.2\textwidth]{3} \\
\end{center}

\vspace{0.5cm}

Entonces para finalizar este año se visualizará las frecuencias de las termoeléctricas por estado.\\

\vspace{1cm}

\begin{center}
    \includegraphics[width= 1.2\textwidth]{7}\\
    
    \vspace{1cm}

    \underline{Año 2023: Año donde la Guiteras cumple}\\

    \vspace{1cm}
    \includegraphics[width= 1.2\textwidth]{8} \\
    
\end{center}

\vspace{1cm}

Aquí se observa cómo se comportaron los parámetros en el año, en lo que se destaca los MW indisponibles por averías, cerca del mes de abril hasta junio mas o menos. Esto se debe de confirmar que durante ese tiempo hubo un número elevando de termoeléctricas fuera de servicio\\

\vspace{0.5cm}
\begin{center}
   \includegraphics[width= 1.2\textwidth]{9}\\ 
\end{center}

\vspace{0.5cm}

Lo que ocurrió en esas fechas fue que se reporta un derrumbe de un tabique en el área de la chimenea de la Central Termoeléctrica Antonio Guiteras.\\

\vspace{0.5cm}

En junio la unidad sufre ahora las secuelas de un fallo en uno de los tubos de la caldera.\\
En agosto la termoeléctrica Guiteras sale del SEN por sobreconsumo de agua y otros defectos.\\
Al siguiente mes un cambio radical en el pronóstico de déficit de generación provocó la salida del sistema de la termoeléctrica Antonio Guiteras, a causa de una nueva avería, dejando, al menos, 12 horas de apagones en la última jornada. Septiembre queda registrado como el mes en donde hubo un mayor déficit en todo el año.\\

\vspace{0.5cm}
\begin{center}
    \includegraphics[width=1.2\textwidth]{10}
\end{center}

\vspace{0.5cm}

Por lo que varias unidades de la Empresa Eléctrica reportaron largos apagones para los próximos días.\\
En esa misma semana la caldera de la Guiteras presenta desperfectos, por lo que es forzada su parada. \\
En noviembre, el gobierno de Matanzas anuncia que la Guiteras cumple con su plan de producción. Según explican se han generado 1 139 408 MW, al parecer un número que ellos consideran elevadísimo y que se ha logrado pese a que la Guiteras ha estado sin funcionar hasta por meses.(www.CubaDebate.cu)\\
Tantas dificultades pero de todas se salió, así que la Guiteras se viste de VICTORIA y junto a ella sus trabajadores.\\
Para finalizar se muestra de forma general cómo se comportaron las termoeléctricas en este año cuando estaban fuera de servicio o en mantenimiento.\\

\vspace{0.5cm}
\begin{center}
    \includegraphics[width= 1.2\textwidth]{11}
\end{center}



\end{document}