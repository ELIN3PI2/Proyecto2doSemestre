\documentclass{beamer}
\usepackage{graphicx}

\usetheme{Madrid}
% Title page details: 
\title{Infraestructura del Sistema Energético} 
\subtitle{Un análisis detallado de su funcionamiento}
\author{Equipo B}

\begin{document}

% Title page frame
\begin{frame}
    \titlepage 
\end{frame}


\section{Introducción}
\begin{frame}{Introducción}
    La situación energética en Cuba en los últimos años ha sido compleja.\\
    \vspace{0.5cm}
    Factores que agravan la situación:\\
    \begin{itemize}
        \item El envejecimiento de la infraestructura eléctrica
        \item La escasez de combustible
        \item Las limitaciones en la capacidad de generación
        \item Los impactos de fenómenos climáticos extremos
    \end{itemize}
\end{frame}

\section{Data Product}
\begin{frame}{Data Product}
    \begin{block}{...}
        ¿Cómo el usuario puede familiarizarse con el tema?
    \end{block}
    \vspace{1cm}
   \begin{figure}[h]
  \centering
  \includegraphics[width=0.5\textwidth]{A.png}
\end{figure}
\end{frame} 
\centering

    \begin{frame}{Algunas de las visualizaciones disponibles}
     \begin{figure}[h]
    \centering
    \includegraphics[width=0.6\textwidth]{C.png}
    \end{figure}
    \vspace{0.5cm}
    \begin{figure}[h]
    \centering
    \includegraphics[width=0.6\textwidth]{D.png}
    \end{figure}
    \end{frame}
   
    
    \begin{frame}{Disponibilidad vs Demanda}
    \centering
    \includegraphics[scale=0.4]{E}
    \end{frame}
   

    \begin{frame}{MW limitados en la generación térmica}
    \includegraphics[scale=0.3]{F.png}\\
    \vspace{1cm}
    \newline 
    MW indisponibles por averías
    \includegraphics[scale=0.3]{G.png}\\
    \end{frame}

    \begin{frame}{MW indisponibles por avería }
    \includegraphics[scale=0.3]{H.png}\\
    \vspace{1cm}
    \includegraphics[scale=0.3]{I.png}\\
    \end{frame}

    \begin{frame}{MW indisponibles por avería}
    \begin{figure}[h]
    \centering
    \includegraphics[width=0.6\textwidth]{J.png}
    \end{figure}
    \end{frame}
    
    \begin{frame}{Termoeléctricas fuera de servicio y en mantenimiento}
        \begin{figure}[h]
        \centering
        \includegraphics[width=0.6\textwidth]{K.png}
        \end{figure}
    \end{frame}
   
    \section{Informe}
    \begin{frame}{Informe}
        \begin{block}{...}
            ¿Qué ocurre en las termoeléctricas cubanas?
        \end{block}
         \begin{figure}[h]
        \centering
        \includegraphics[width=0.9\textwidth]{L.png}
        \end{figure}
    \end{frame}
\section{Podcast}
\begin{frame}{Podcast}
     \begin{block}{...}
            Objetivo:
        \end{block}
    Que las personas tengan un mayor conocimiento sobre la energía fotovoltaica y cómo puede ayudar esto a contrarrestar las afectaciones eléctricas.
\end{frame}

\section{Conclusiones}
\begin{frame}{Conclusiones}
    \centering
    \begin{itemize}
        \item Comportamiento de varios parámetros como la máxima afectación durante el horario pico, la disponibilidad y la demanda, las termoeléctricas fuera de servicio, entre otros
        \item Impacto en el suministro energético
        \item Necesidad de inversión y mantenimiento
        \end{itemize}
    \vspace{3cm}
 \textbf{Muchas Gracias}
\end{frame}


\end{document}